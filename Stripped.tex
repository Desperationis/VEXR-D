\section{[Complete] Removing Stripped Screws}

\begin{figure}[h]
    \centering
    \includegraphics[width=\textwidth,height=8.5cm,keepaspectratio=true]{Stripped/1}
    \includegraphics[width=\textwidth,height=8.5cm,keepaspectratio=true]{Stripped/2}
    \caption{
        (Left) The stripped screws that used to be in the shaft collars. (Right) The torx bit that removed them.
    }
\end{figure}

Once upon a time, an idiot overtightened two shaft collars and essentially fused them to two axles as they had become stripped; Those two axles remained useless for months. Normally, you remove stripped screws by either cutting a dent in them so that a flat head screwdriver can loosen it or by using a rubber band to increase traction. However, as both screws were deep inside the shaft collar, neither method would've worked.

Due to this, I tried using a slightly oversized torx bit to turn the screw on the shaft collar with; Torx was used as it had this really sharp edges that would be able to catch the remains of what used to be a hexagonal hole. After a bit of trial and error with many torx bits, I was able to remove the screw completely with great ease on BOTH collars with the SAME bit.

With this, I propose we use a series of torx bits to remove stripped screws with. The sharp edges of the torx bit catch the grooves well, spreads force evenly, and can actually be used as a replacement for hex wrenches should we lose them.

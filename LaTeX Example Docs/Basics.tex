\section{The Basics}
Before learning tabout the crazy stuff in \LaTeX{}, you need to get the basics down first, meaning you need to know how commands, environments, text, and subsections work.

\subsection{Commands}
All commands in latex start with a backslash, such as the \texttt{\textbackslash textbackslash}, \texttt{\textbackslash section}, and \texttt{\textbackslash title} commands. Most commands have \textbf{parameters} you can pass in by using enclosed curly brackets with a command, like so:

\begin{center}
\texttt{\textbackslash command\{\textit{parameter}\}}
\end{center}

Commands use parameters in different ways, though nearly all take in a string of letters:

\begin{description}
\centering
\item[\texttt{\textbackslash texttt}] Any sentence that you want to be in typewrite font.
\item[\texttt{\textbackslash begin}] The name of a special environment you want to begin. 
\item[\texttt{\textbackslash end}] The name of a special environment you want to switch out of. 
\end{description}

Sometimes, commands need multiple arguments. This is troublesome in \LaTeX, as comma-separated parameters can result in accidental errors in sentences with commas. Due to this, a second pair of curly brackets is all that is needed: 

\begin{center}
\texttt{\textbackslash command\{\textit{parameter 1}\}\{\textit{parameter 2}\}}
\end{center}

(Keep in mind however, that some arguments do use comma-separated arguments if they do not take in an entire sentence, such as \texttt{\textbackslash includeonly}.)\\

Commands can also have \textbf{optional parameters}. When calling a command, these options can be accessed via brackets, like so:

\begin{center}
\texttt{\textbackslash command[\textit{optional parameter}]\{\textit{parameter}\}}
\end{center}

And, like regular parameters, can be used for multiple arguments like so:
 
\begin{center}
\texttt{\textbackslash command[\textit{optional parameter 1}][\textit{optional parameter 2}]\{\textit{parameter}\}}
\end{center}

Finally, you can also create commands using \texttt{\textbackslash newcommand}. This command accepts two parameters: \textit{name} and \textit{function}, where \textit{name} is the name of the command (with the backslash) and \textit{function} are the commands you execute whenever the command is called. 

\begin{center}
\texttt{\textbackslash newcommand\{\textit{name}\}\{\textit{function}\}}
\end{center}

When creating a new command, you can also specify the number of parameters and use them in your code. Here's an example. 

\begin{center}
\texttt{\textbackslash newcommand[2]\{\textbackslash myCommand\}\{Section \textbf{\#1}\}}
\end{center}

Let's analyze this command. The optional parameter \texttt{2} first tells the command the number of parameters this command can take. Then, in the function parameter, \textbf{\#1} tells the command to directly insert the first parameter of this command in the function parameter, thereby making parameter \texttt{\#2} useless. Finally, \texttt{\textbackslash myCommand} is the name of the command. 

You can also specify optional parameters like so: 
\begin{center}
\texttt{\textbackslash newcommand[2][hey]\{\textbackslash myCommand\}\{Section \textbf{\#1}\}}
\end{center}

... where the additional closed brackets after the first are the default values of the first parameter, turning it into an optional parameter. 




\subsection{Environments}
In \LaTeX, environments are simply blocks of formatted text (You can think of them as miniature programs that format the text inside of them in a specific way). To start an environment, you simply type \texttt{\textbackslash begin}\{\textit{name}\}, where \textit{name} is the specific name of a defined environment. After that, you type \texttt{\textbackslash end}\{\textit{name}\}, which ends an environment. Any text between these two commands will be formatted according to the rules of the environment, and is usually delimited by \texttt{\textbackslash n}, or \texttt{RETURN}. 

To create your own enviornments, you can use the \texttt{\textbackslash newenvironment} command. This command takes in \textbf{four or more} parameters:

\begin{center}
\texttt{\textbackslash newenvironment\{\textit{name}\}[\textit{\# of parameters}][\textit{...}]\{\textit{\textbackslash begin commands}\}\{\textit{\textbackslash end commands}\}}
\end{center}

... where \texttt{\textit{name}} is the name of the environment, \texttt{\textit{\# of parameters}} are the number of parameters required in the environment, \texttt{\textit{...}} being the default values of optional parameters, \texttt{\textit{\textbackslash begin commands}} being the commands called at the beginning of the environment, and \texttt{\textit{\textbackslash end commands}} being the commands called at the end of the environment.





\subsection{Sections}
Sections are simply commands that insert a line of formatted text given a parameter. The commands \texttt{\textbackslash section}, \texttt{\textbackslash subsection}, and \texttt{\textbackslash subsubsection} all produce a formatted line of varying size given a sentence as an argument. 

However, those commands assume you want \textbf{numbered} sections that can be visible in \texttt{\textbackslash tableofcontents}. If you want \textbf{unnumbered} sections, then you simply add a * after each command. E.x. \texttt{\textbackslash section}*\{\} (These sections are not tracked by \texttt{\textbackslash tableofcontents}, though).



\subsection{Text Magic}
All text can be manipulated by putting it as a parameter through a variety of commands. Here are a few common ones:

\begin{description}
\centering
\item[\texttt{\textbackslash textbf}] \textbf{Bolds} a line of text. 
\item[\texttt{\textbackslash textit}] \textit{Italizes} a line of text. 
\item[\texttt{\textbackslash textsc}] Writes a line of text in \textsc{small caps}. 
\item[\texttt{\textbackslash texttt}] Writes a line of text in \texttt{typewritter style}. 
\item[\texttt{\textbackslash emph}] \emph{Emphasizes} a line of text. This makes it \emph{italicized} with normal text, \textit{or \emph{normal} in italized text}.
\end{description} 


However, some characters, such as \textbackslash, \#, \$, \{, and \} cannot be inputed normally as they are \textbf{reserved characters}. To get around this, you use the following commands:

\begin{center}
\begin{tabular}{ccc}
\texttt{\textbackslash \&} = \&  & \texttt{\textbackslash \#} = \#  & \texttt{\textbackslash \$} = \$  \\
\texttt{\textbackslash \$} = \$  & \texttt{\textbackslash \%} = \%  & \texttt{\textbackslash \{} = \{ \\
\texttt{\textbackslash \}} = \} & \texttt{\textbackslash \_} = \_ & \texttt{\textbackslash textasciicircum} = \textasciicircum \\
\texttt{\textbackslash textasciitilde} = \textasciitilde & \texttt{\textbackslash textbackslash} = \textbackslash
\end{tabular}
\end{center}

Quotes have an exception as well. To make a quote, you use two marks (``) and two single quotes ("). You can also use a single mark(`) and a single quote('), if you'd like. 

Finally, line breaks can be achieved by writing down \texttt{\textbackslash \textbackslash} anywhere in \TeX{} code. When a line break is written, a new paragraph is formed. 





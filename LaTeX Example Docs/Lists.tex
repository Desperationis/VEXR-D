\section{Lists and Footnotes}
\subsection{Lists}
\textbf{Lists} are a group of items. In \LaTeX, there are two main types of lists: \textbf{numbered} and \textbf{unnumbered}. To get a \textbf{numbered} list, you use the \texttt{enumerate} environment:\\

\begin{center}
\begin{tabular}{cc}

Code & Result\\
\hline\\

\parbox{4cm}{
	\texttt{\textbackslash begin\{enumerate\}\\
	\textbackslash item something\\
	\textbackslash item something else\\
	\textbackslash item something more\\
	\textbackslash end\{enumerate\}\\
	}
} 

&

\parbox{4cm}{
	\begin{enumerate}
	\item something
	\item something else
	\item something more
	\end{enumerate}
}

\end{tabular}
\end{center}

And to get a \textbf{unnumbered} list, you use \texttt{itemize}:\\

\begin{center}
\begin{tabular}{cc}

Code & Result\\
\hline\\

\parbox{4cm}{
	\texttt{\textbackslash begin\{itemize\}\\
	\textbackslash item something\\
	\textbackslash item something else\\
	\textbackslash item something more\\
	\textbackslash end\{itemize\}\\
	}
} 

&

\parbox{4cm}{
	\begin{itemize}
	\item something
	\item something else
	\item something more
	\end{itemize}
}

\end{tabular}
\end{center}


\textit{Any} type of list can be \textbf{nested} into one another as well:\\

\begin{center}
\begin{tabular}{cc}

Code & Result\\
\hline\\

\parbox{4cm}{
	\texttt{\textbackslash begin\{enumerate\}\\
	\textbackslash item something\\
	\textbackslash begin\{itemize\}\\
	\textbackslash item something\\
	\textbackslash item something else\\
	\textbackslash item something more\\
	\textbackslash end\{itemize\}\\
	\textbackslash item something else\\
	\textbackslash item something more\\
	\textbackslash end\{enumerate\}\\
	}
} 

&

\parbox{4cm}{
	\begin{enumerate}
	\item something
		\begin{itemize}
		\item something
		\item something else
		\item something more
		\end{itemize}

	\item something else
	\item something more
	\end{enumerate}
}
\end{tabular}
\end{center}


Finally, there exists \textbf{description} lists, which are handy when listing definitions:\\

\begin{center}
\begin{tabular}{cc}

Code & Result\\
\hline\\

\parbox{6cm}{
	\texttt{\textbackslash begin\{description\}\\
	\textbackslash item[term1]  something\\
	\textbackslash item[term2]  something else\\
	\textbackslash item[term3]  something more\\
	\textbackslash end\{description\}\\
	}
} 

&

\parbox{4cm}{
	\begin{description}
	\item[term1] something
	\item[term2] something else
	\item[term3] something more
	\end{description}
}

\end{tabular}
\end{center}


Like previous lists, description lists can be \textbf{nested} as well:\\

\begin{center}
\begin{tabular}{cc}

Code & Result\\
\hline\\

\parbox{6cm}{
	\texttt{\textbackslash begin\{enumerate\}\\
	\textbackslash item something\\
	\textbackslash begin\{description\}\\
	\textbackslash item[term1]  something\\
	\textbackslash item[term2]  something else\\
	\textbackslash item[term3]  something more\\
	\textbackslash end\{description\}\\
	\textbackslash item something else\\
	\textbackslash item something more\\
	\textbackslash end\{enumerate\}\\
	}
} 

&

\parbox{6cm}{
	\begin{enumerate}
	\item something
		\begin{description}
		\item[term1] something
		\item[term2] something else
		\item[term3] something more
		\end{description}

	\item something else
	\item something more
	\end{enumerate}
}
\end{tabular}
\end{center}






\subsection{Footnotes}
\textbf{Footnotes} are incredibly easy to add anywhere. Simply use \texttt{\textbackslash footnote} to add any sentence (its argument) as a footnote \footnote{Like this}.



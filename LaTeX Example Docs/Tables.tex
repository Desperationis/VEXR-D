\section{Tables}
\textbf{Tables} in \LaTeX{} can be accomplished through the use of a special environment called \texttt{tabular}. However, unlike other basic environments, \texttt{tabular} takes in a parameter that both tells how each column will be justified and the amount of columns. Here's an example:\\

\begin{center}
\begin{tabular}{cc}

Code & Result\\
\hline\\

\parbox{4cm} {
    \texttt{\textbackslash begin\{tabular\}\{ccc\}\\
    0, 0 \& 1, 0 \& 2, 0 \textbackslash \textbackslash \\
    0, 1 \& 1, 1 \& 2, 1 \textbackslash \textbackslash \\
    0, 2 \& 1, 2 \& 2, 2 \textbackslash \textbackslash \\
    \textbackslash end\{tabular\}\\
    }
}

&

\begin{tabular}{ccc}
    0, 0 & 1, 0 & 2, 0 \\
    0, 1 & 1, 1 & 2, 1 \\
    0, 2 & 1, 2 & 2, 2 \\
\end{tabular}

\end{tabular}
\end{center}

The code above shows
